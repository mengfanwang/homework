\documentclass[22pt]{article} 
\usepackage{geometry} 
\usepackage{float} 
\usepackage{graphicx}
\usepackage{caption}
\usepackage{subfigure}
\usepackage{amsmath}
\usepackage{array}
\usepackage{amsfonts}
\geometry{left=2.0cm,right=2.0cm,top=0.5cm,bottom=2cm}
	\author{Mengfan Wang} 
	\title{Stochastic Signal Systems Homework 4} 
\begin{document}
		\maketitle 
	\paragraph{1}
	\begin{align}
	E[X|X>0] & = \int_{-\infty}^{\infty} xf(x|X>0)dx \\ 
	& = \int_{0}^{\infty} \frac{2x}{\sqrt{2 \pi}\sigma}exp(-\frac{x^2}{2 \sigma^2})dx\\
	& = -\frac{2 \sigma}{\sqrt{2 \pi}}exp(-\frac{x^2}{2 \sigma^2})|_{0}^{\infty}\\
	& = \frac{2 \sigma}{\sqrt{2 \pi}}
	\end{align}
	\begin{align}
	E[X^2|X>0] & = \int_{-\infty}^{\infty} x^2f(x|X>0)dx \\ 
	& = \int_{0}^{\infty} \frac{2x^2}{\sqrt{2 \pi}\sigma}exp(-\frac{x^2}{2 \sigma^2})dx \\
	& = -\int_{0}^{\infty} \frac{2x \sigma}{\sqrt{2 \pi}}d(exp(-\frac{x^2}{2 \sigma^2})) \\
	& = -\frac{2x \sigma}{\sqrt{2 \pi}}exp(-\frac{x^2}{2 \sigma^2})|_{0}^{\infty} + \int_{0}^{\infty} \frac{2\sigma}{\sqrt{2 \pi}}exp(-\frac{x^2}{2 \sigma^2})dx \\
	& = \sigma^2
	\end{align}
	So, $VAR[X|X>0] = E[X^2|X>0] - E[X|X>0]^2 = \sigma^2 - \frac{2 \sigma^2}{\pi} = \frac{\pi-2}{\pi}\sigma^2$.
    
    \paragraph{2}
    For a Gaussian distribution, we know its characteristic function is $\Phi_X(\omega) = exp(j \omega m- \frac{1}{2}\sigma^2 \omega^2)$.
    So,
    \begin{align}
    E[g(x)] & = \int_{-\infty}^{\infty} f(x)\cos x dx\\
 	& = \frac{1}{2}\int_{-\infty}^{\infty} f(x)(e^{jx}+e^{-jx})		dx\\
 	& = \frac{1}{2}(\Phi_X(1)+\Phi_X(-1))\\
 	& = \frac{1}{2}(exp(jm- \frac{1}{2}\sigma^2)+ \exp(-jm- \frac{1}{2}\sigma^2))\\
 	& = \cos m \exp(-\frac{1}{2}\sigma^2)
    \end{align}
    \paragraph{3} 
    \begin{align}
    \Phi_X(\omega) = \int_{-b}^{b}\frac{e^{j \omega x}}{2b} dx = \frac{e^{j \omega b}-e^{-j \omega b}}{2j \omega b}
    \end{align}
    \begin{align}
    E[X]&= -j\frac{d}{d \omega}\Phi_X(\omega)|_{\omega = 0} \\
    & = -j\frac{j \omega b e^{j \omega b}+j \omega be^{-j \omega b}-e^{j \omega b}+e^{-j \omega b}}{2j \omega^2 b}|_{\omega=0}\\
    & = -j\lim_{\omega \rightarrow 0}\frac{(j \omega b e^{j \omega b}+j \omega be^{-j \omega b}-e^{j \omega b}+e^{-j \omega b})''}{(2j \omega^2 b)''}\\
    & = 0
    \end{align}


    \paragraph{4}
    \begin{align}
    E[g(X)] & = \int_{-\infty}^{\infty} f(x)g(x) dx\\
    & = \int_{0}^{\infty} e^{-x}\frac{\sin x}{x}  dx\\
    & = \int_{0}^{\infty} e^{-x}\frac{1}{2}\int_{-1}^{1}e^{jux} dudx\\
    & = \int_{-1}^{1}\int_{0}^{\infty}\frac{1}{2}e^{-x}e^{jux}dx du\\
    & = \int_{-1}^{1} \frac{1}{2(ju-1)}e^{jux-x}|^{\infty}_{0} du\\
    & = -\int_{-1}^{1} \frac{1}{2(ju-1)} du\\
    & = - \frac{1}{2j}(ln|2j-2|-ln|-2j-2|)\\
    & = - \frac{1}{2j}(ln2\sqrt{2} + \frac{3}{4}j \pi -ln2\sqrt{2} - \frac{5}{4}j \pi)\\
    & = \frac{\pi}{4		}
    \end{align}

    \paragraph{5}
    \begin{align}
    f_X(x) & = \int_{0}^{\infty} xe^{-x(1+y)}dy\\
    &  = -e^{-x(1+y)}|_0^{\infty}\\
    & = e^{-x}
    \end{align}
    So,\begin{align}
    F_X(x) = 
				\begin{cases}
				e^{-x} & x >0 \\
				0 & x \leq 0 
				\end{cases}
	\end{align}
    \begin{align}
    f_Y(y) & = \int_{0}^{\infty} xe^{-x(1+y)}dx\\
    & = \int_{0}^{\infty} -\frac{x}{1+y} de^{-x(1+y)}\\
    & = -\frac{x}{1+y} e^{-x(1+y)}|_0^{\infty} + \int_{0}^{\infty}\frac{1}{1+y} e^{-x(1+y)} dx\\
    & = -\frac{1}{(1+y)^2}e^{-x(1+y)}|_0^{\infty} \\
    & = \frac{1}{(1+y)^2}
    \end{align}
    So,\begin{align}
    F_Y(y) = 
				\begin{cases}
				\frac{1}{(1+y)^2} & y > 0\\
				0 & y \leq 0
				\end{cases}
	\end{align}


    \paragraph{6}
    According to the conditions, $f_{X,Y}(x,y) = \frac{1}{2 \pi	\sigma^2}exp(-\frac{x^2+y^2}{2 \sigma^2}) $. 
    Suppose $x = r\cos \theta$, $y = r\sin \theta$:
    \begin{align}
    P[X^2 + Y^2 <q^2] & = \frac{1}{2 \pi	\sigma^2} \int_{0}^{2 \pi}\int_{0}^{q}exp(-\frac{r^2}{2 \sigma^2}) rdrd \theta\\
    & = 	\frac{1}{2 \pi	\sigma^2} \int_{0}^{2 \pi} \sigma^2 (1 - exp(-\frac{q^2}{2 \sigma^2})) d \theta	\\
    & = 1 - exp(-\frac{q^2}{2 \sigma^2})
    \end{align}
   \paragraph{7}
   		\subparagraph{a} 
   		\begin{align}
   		F_{X,Y}(\pi/2,\pi/2) & = c\int_{0}^{\pi/2}\int_{0}^{\pi/2}sin(x+y)dxdy\\
   		& = c\int_{0}^{\pi/2}cosy-cos(\pi/2+y)dx\\
   		& = 2c = 1
   		\end{align}
   		So $c = 1/2.$
   		\subparagraph{b}
   		When $x<0 $ or $y<0$, $F_{X,Y}(x,y) = 0$; When $x > \pi/2, y> \pi	/2$, $F_{X,Y}(x,y) = 1$.
   		When $0\leq x \leq \pi/2$ and $0\leq y \leq \pi	/2$,
   		\begin{align}
   		F_{X,Y}(x,y) & = \frac{1}{2}\int_{0}^{x}\int_{0}^{y} sin(x'+y') dy'dx' \\
   		& = \frac{1}{2}\int_{0}^{x} cosx' - cos(x'+y) dx'\\
   		& = \frac{1}{2}(sinx + siny - sin(x+y))
   		\end{align}
   		When $0\leq x \leq \pi/2$ but $y > \pi	/2$:
   		\begin{align}
   		F_{X,Y}(x,y) & = \frac{1}{2}\int_{0}^{\pi	/2}\int_{0}^{x} sin(x'+y') dx'dy' \\
   		& =\frac{1}{2} \int_{0}^{\pi	/2} cosy' - cos(x+y') dy'\\
   		& =\frac{1}{2}( 1 + sinx - sin(x+\pi/2))
   		\end{align}
   		Similarly, when $0\leq y \leq \pi/2$ but $x > \pi	/2$, we have $F_{X,Y}(x,y) =\frac{1}{2}( 1 + siny - sin(y+\pi/2)) $
   		So,\begin{align}
    		F_{X,Y}(x,y) = 
				\begin{cases}
				0 & x<0/y<0\\
				\frac{1}{2}(sinx + siny - sin(x+y) ) & 0\leq x \leq \pi/2 \& 0\leq y \leq \pi/2\\
				\frac{1}{2}(1 + sinx - sin(x+\pi/2) )& 0\leq x \leq \pi/2 \& y > \pi	/2\\
				\frac{1}{2}(1 + siny - sin(y+\pi/2)) & x > \pi	/2 \&  0\leq y \leq \pi/2\\
				1 & x > \pi	/2  \& y > \pi	/2 
				\end{cases}
		\end{align}

		\subparagraph{c}
		When $0\leq x \leq \pi/2 $:
		\begin{align}
		f_X(x) & = \frac{1}{2}\int_{0}^{\pi/2}sin(x+y')dy\\
		& = \frac{1}{2}(cosx - cos(x+\pi/2))
		\end{align}
	    	So,\begin{align}
    f_X(x) = 
				\begin{cases}
				\frac{1}{2}(cosx - cos(x+\pi/2))& 0\leq x \leq \pi/2 \\
				0 & otherwise
				\end{cases}
	\end{align}
	 Similarly,\begin{align}
    f_Y(y) = 
				\begin{cases}
				\frac{1}{2}(cosy - cos(y+\pi/2))& 0\leq y \leq \pi/2 \\
				0 & otherwise
				\end{cases}
	\end{align}

	\paragraph{8}		
	 \subparagraph{a} 
	 \begin{align}
	 f_X(x) & = \int_{0}^{1}k(x+y)dy\\
	 & = k( xy + \frac{1}{2}y^2)|^1_0 \\
	 & = k(x+ \frac{1}{2})
	 \end{align}
	 Similarly, $f_Y(y) = k(y+ \frac{1}{2})$.
	 So, $f_{X,Y}(x,y) \not= f_X(x)f_Y(y)$, they are not independent.		

	 \subparagraph{b}
	 \begin{align}
	 f_Y(y|x) & = \frac{f_{X,Y}(x,y)}{f_X(x)}\\
	 & = \frac{k(x+y)}{k(x+\frac{1}{2})}\\
	 & = \frac{x+y}{x+\frac{1}{2}}
		 \end{align}

	 \paragraph{9}
	 Firstly, we have
	 \begin{align}
	 f_X(x) =  \int_{x}^{\infty}e^{-y}dy = e^{-x}
	 \end{align}
	 and
	 \begin{align}
	 f_Y(y) = \int_{0}^{y}e^{-y}dx = ye^{-y}
	 \end{align}
	 So,
	 \begin{align}
	 E[X|y] & =  \int_{0}^{y} x \frac{f_{X,Y}(x,y)}{f_Y(y)} dx\\
	 & = \int_{0		}^{y} \frac{x}{y} dx\\
	 & = \frac{y}{2}
	 \end{align}
	 \begin{align}
	 E[Y|x] & = \int_{x}^{\infty} y\frac{f_{X,Y}(x,y)}{f_X(x)} dy\\
	 & = \int_{x}^{\infty} y e^{x-y} dy\\
	 & = \int_{x}^{\infty} -y de^{x-y}\\
	 & = -ye^{x-y}|_{x}^{\infty} +  \int_{x}^{\infty} e^{x-y}dy\\
	 & = x+1
	 \end{align}

	 \paragraph{10}
	 \begin{align}
	 f_Y(y) & =  \int_{0}^{1} 6(1-x-y)dx\\
	 & = 3(1-y)^2 
	 \end{align}
	 \begin{align}
	 E[X|y] & = \int_{0}^{1-y} x\frac{f_{X,Y}(x,y)}{f_Y(y)} dx\\
	 & = \int_{0}^{1-y} x\frac{2(1-x-y)}{(1-y)^2} dx\\
 	 & = \frac{1}{3}(1-y)
	 \end{align}
	 \begin{align}
	 E[X^2|y] & = \int_{0}^{1-y} x^2\frac{f_{X,Y}(x,y)}{f_Y(y)} dx\\
	 & = \int_{0}^{1-y} x^2\frac{2(1-x-y)}{(1-y)^2} dx\\
 	 & = \frac{1}{6}(1-y)^2
	 \end{align}

	 \paragraph{11}
	 \begin{align}
	 P[X<Y] & = \int_{0}^{1}\int_{0}^{y}e^{-x}dxdy\\
	 & = \int_{0}^{1}1-e^{-y}dy\\
	 & = e^{-1}	
	 \end{align}
	 So, $P[X\geq Y] = 1- P[X<Y] = 1 - e^{-1}$.


\end{document}